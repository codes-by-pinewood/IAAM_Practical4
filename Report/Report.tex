\documentclass{article}

% Language setting
% Replace `english' with e.g. `spanish' to change the document language
\usepackage[english]{babel}

% Set page size and margins
% Replace `letterpaper' with `a4paper' for UK/EU standard size
\usepackage[letterpaper,top=2cm,bottom=2cm,left=3cm,right=3cm,marginparwidth=1.75cm]{geometry}
\usepackage{algorithm}
\usepackage{algpseudocode}
\usepackage{amsmath}
\usepackage{graphicx}
\usepackage{caption,subcaption}
\usepackage[colorlinks=true, allcolors=blue]{hyperref}

\title{Image Analysis Assignment 4}
\author{Sherry Usman, Megan Mirnalini Sundaram R}
\begin{document}
\maketitle

\section{Question 4.1}
\subsection{Choice of Cells}

\subsection{Manual Check of the tracing of 5 cells}

\subsection{Algorithm to trace the cells}
In order to track the movement of the cells, it is essential that the images must be processed first, to remove the glare and other distortions from the setup. Hence, the first part of the algorithm deals with the processing and labeling of the cells in the image. This allows us to gather further details on the cells, which were essential to tracing the cells. 

\begin{algorithm}[h!]
\caption{Tracing the Cells}\label{cell-trace}
\begin{algorithmic}[1]
\Procedure{TracingCells}{$\text{images}$}
    \State $\text{closest\_row[figure]} \gets \text{empty list}$
    \For{$\text{image}$ \textbf{in} $\text{images}$}
        \State $gray\_image \gets \text{Grayscale}(image)$
        \State $image\_contrasted \gets \text{ContrastStretch}(image)$
        \State $image\_eroded \gets \text{Erosion}(image\_contrasted)$
        \State $threshold\_image \gets \text{RangeThreshold}(mage\_eroded)$
        \State $image\_labeled \gets \text{Label}(threshold\_image)$
        \State Determine the center, size, and mean of each image
    \EndFor
    \State Compile the measurements into a dataframe
    \For{$\text{image}$ \textbf{in} $\text{images}$}
        \State $matching\_points \gets \text{center\_of\_gravity}(image) \And \text{center\_of\_gravity}(image+1)$
        \State $filtered\_matching\_points \gets \text{center\_of\_gravity}(image) \And \text{center\_of\_gravity}(image+1)$
        \State Superimpose points on images
    \EndFor
    \State \textbf{return} Graph of points on images
\EndProcedure
\end{algorithmic}
\end{algorithm}

\par The processing of the images is first done by converting the image to a grayscale, and then contrast stretching with a lower bound of 0 and an upper bound of 75. The contrast-stretched image was then subjected to erosion. This removed the white noise on the image and highlighted the cells of interest. The image was then, thresholded and labeled. This allowed us to get the measurements of the objects i.e., cells such as center of gravity, size and mean of the objects. 
\par The measurements were then compiled into a dataframe. The center of gravity of each object allowed us to track the movement in each image, as this served as the x- and y-coordinates for the movement. The Euclidean distance
\subsection{Application of Algorithm and Results}

\section{Part 4.2}
\subsection{Shape and Texture}
\subsection{Cell Velocity and Distance Trajectory}
\subsection{Differences in Trajectory}
\subsection{Differences in Conditions}
\subsection{Correlation between speed with shape and texture}
\bibliographystyle{alpha}
\bibliography{sample}

\end{document}